\makeatletter
\def\thickhrulefill{\leavevmode \leaders \hrule height 1ex \hfill \kern \z@}
\def\@makechapterhead#1{%
  \vspace*{10\p@}%
  {\parindent \z@ \centering \reset@font
        {\Huge \scshape \thechapter}
        \par\nobreak
        \vspace*{15\p@}%
        \interlinepenalty\@M
        \begin{tabular}{@{\qquad}c@{\qquad}}
          \hline
          \\
          {\Huge \bfseries #1\par\nobreak} \\
          \\
          \hline
        \end{tabular}
    \vskip 100\p@
  }}
\def\@makeschapterhead#1{%
  \vspace*{10\p@}%
  {\parindent \z@ \centering \reset@font
        {\Huge \scshape \vphantom{\thechapter}}
        \par\nobreak
        \vspace*{15\p@}%
        \interlinepenalty\@M
        \begin{tabular}{@{\qquad}c@{\qquad}}
          \hline
          \\
          {\Huge \bfseries #1\par\nobreak} \\
          \\
          \hline
        \end{tabular}
    \vskip 100\p@
  }}

\chapter{Bevezetés}
  A térfigyelő kamerák egyre nagyobb figyelmet igényelnek a mindennapi életünkben. Kamerák nyomon követnek és rögzítenek eseményeket, amelyeket megfigyelhetünk, esetleg később megtekinthetjük.
  
  
  Azonban nem mindig tudjuk, hogy az adott kamera a térben pontosan hol helyezkedik, milyen állapotban van, és mekkora a látómezője. Ha térkép alapokra tudnánk helyezni egy kamerával kapcsolatos térbeli adatait, akkor sokkal könnyebben lehetne nyomonkövetni állapotukat karbantartási szempontból, helyzet változtatásukról, ezáltal könnyebben tudnánk tájékoztatni a felhasználóit.
